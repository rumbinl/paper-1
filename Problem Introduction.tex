\section{Problem Introduction}
Pacman is a video game which utilizes individual textures of 8x8 pixels to render its core components as can be seen in Figure~\ref{PacmanLevelGrid}, each respectively represented by one square in the dotted grid.

\begin{figure}[H]
\centering
\includegraphics[width=0.4\linewidth]{Image-1.jpg}
\caption {Pacman level divided into squares of 8x8 pixels.\autocite{pittman_pac-man_2009}}\label{PacmanLevelGrid}
\end{figure}

Since the pixel length of a wall is 1 pixel, the wall textures all contain an asymmetry over the central vertical line as seen in Figure~\ref{StraightWallTexture} and~\ref{CornerTexture}. The asymmetry is the basis of the visual appearance of levels which are formed by an oriented placing of these textures. 

\begin{figure}[H]
\centering
\includegraphics[width=0.4\linewidth]{Image-2.png}
\caption {A single wall texture. As can be seen this wall is clamped to the right, and would connect a tile above it and below it.\autocite{myself}}\label{StraightWallTexture}
\end{figure}

\begin{figure}[H]
\centering
\includegraphics[width=0.4\linewidth]{Image-3.png}
\caption {Square texture of a corner.\autocite{myself}}\label{CornerTexture}
\end{figure}

Because of this asymmetry, it must be decided in what direction walls are clamped, that is in what direction the asymetry should act to ensure visual consistency. When all walls are clamped in the same direction we receive unsatisfactory results as seen on the left part of Figure~\ref{WallTextureAsymmetry}. 

\begin{figure}[H]
\centering
\includegraphics[width=0.4\linewidth]{Image-4.png}
\caption {Left - Incorrect orientation of tiles. Right - Correct orientation with walls clamped to appropriate direction. Walls on the left are clamped inward, to the right and walls on the right are clamped inwards to the left. The problem consists of determining which direction points "inwards" for each tile.\autocite{myself}}\label{WallTextureAsymmetry}
\end{figure}

\begin{figure}[H]
\centering
\includegraphics[width=0.8\linewidth]{Image-5.png}
\caption {The algorithm devised through this paper will interpret from the image on the left to the one on the right, reconstructing the visual properties using a simplified outline, only describing the position of the walls.\autocite{myself}}\label{LevelWallConversion}
\end{figure}

This problem will be represented using concepts of graph theory.