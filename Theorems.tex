\section{Properties of Square Lattice Graphs}
\begin{theorem}
Every vertex $v\in V$ of a square lattice graph $G=(V,E)$ representing a closed boundary, must have a degree $d(v)\geq 2$.\label{DegreeLemma}
\end{theorem}
\subsubsection*{Proof of Theorem~\ref{DegreeLemma}}
If a vertex $v\in V$ has degree $d(v)=0$, and the number of vertices $\in V$, $n>1$, then that vertex cannot connect to any other vertices in the square lattice graph, and can thus not form part of the closed boundary. If a vertex $v_0\in V$ has degree $d(v_0)=1$, then any path $p$ where $v\in p$, will not be a Hamiltonian cycle, as the edge $e\in E$ where $v_0\in e$, can only be crossed once, as the adjacent vertex $v_1\in e$, can only be crossed once on path $p$.
\begin{theorem}
In a square lattice graph of a closed boundary wall configuration, in relation to the vertices' grid coordinates, out of the northernmost vertices, the westmost vertex will always be a corner connecting two vertices to its east and its south, respectively.\label{NorthwestCornerLemma}
\end{theorem}
\subsubsection*{Proof of Theorem~\ref{NorthwestCornerLemma}}
In such a lattice graph, we pick the westmost vertex amongst the northernmost vertices. If there is a vertex connected to it, to its north, then our chosen vertex was not the northernmost vertex which is a contradiction. If there is a vertex connected to its west, then it is not the westmost vertex and is again a contradiction. 
\begin{theorem}
A square lattice graph of a closed boundary wall configuration will always have at least 4 corner vertices, each connecting to two vertices, and each pointing into different cardinal directions.\label{FourCornerLemma}
\end{theorem}
\subsubsection*{Proof of Theorem~\ref{FourCornerLemma}}
Following from theorem~\ref{NorthwestCornerLemma}, in a square lattice defining a boundary there is at least one corner corner vertex $v_{se}$ with an edge to its east and one to its south. Since by definition there are no vertices $v\in V$ that are to the north of $v_{se}$, there must be a vertex to its east, directly connected to it by a straight walk, that is a corner vertex $v_{sw}$ with an edge to its south and to its west. If there is a vertex to its right, we can keep walking until there is a vertex that only connects downwards. Next we can walk downwards from this vertex. If vertex that connects eastward is encountered, we follow that walk until there is a walk upwards. The next corner vertex, if on this walk, must have no connections southward, but a connection westward. If however we encounter a node that connects eastward, we must follow it, as per our assumption, every node must be included in the Hamiltonian, similarly we must always continue the walk on the edge going in the opposite direction to find a desired corner vertex. The boundary walk may change directions many times, but since we assume that a Hamiltonian cycle is possible, there must at some point be a vertex connecting to its north and west. This is because the walk  must  reconnect to our first corner vertex, southwards to it which is only possible if at some point the walk deviates to the west after a downward walk. It follows similarly for the last of the four vertices, which points north and east. Note that we used every combination of a pair of a vertical and horizontal cardinal direction once in each of the four corners.$\blacksquare$
