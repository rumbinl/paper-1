\section{Properties of Square Lattice Graphs}
\begin{theorem}
Every vertex $v\in V$ of a square lattice graph $G=(V,E)$ representing a closed boundary, must have a degree $d(v)\geq 2$.\label{DegreeLemma}
\end{theorem}
\subsubsection*{Proof of Theorem~\ref{DegreeLemma}}
If a vertex $v\in V$ has degree $d(v)=0$, and the number of vertices $\in V$, $n>1$, then that vertex cannot connect to any other vertices in the square lattice graph, and can thus not form part of the closed boundary. If a vertex $v_0\in V$ has degree $d(v_0)=1$, then any path $p$ where $v\in p$, will not be a Hamiltonian cycle, as the edge $e\in E$ where $v_0\in e$, can only be crossed once, as the adjacent vertex $v_1\in e$, can only be crossed once on path $p$.
\begin{theorem}
In a square lattice graph of a closed boundary wall configuration, in relation to the vertices' grid coordinates, out of the northernmost vertices, the westmost vertex will always be a corner connecting two vertices to its east and its south, respectively.\label{NorthwestCornerLemma}
\end{theorem}
\subsubsection*{Proof of Theorem~\ref{NorthwestCornerLemma}}
In such a lattice graph, we pick the westmost vertex amongst the northernmost vertices. If there is a vertex connected to it, to its north, then our chosen vertex was not the northernmost vertex which is a contradiction. If there is a vertex connected to its west, then it is not the westmost vertex and is again a contradiction. 
\begin{theorem}
A square lattice graph of a closed boundary wall configuration will always have at least 4 corner vertices, each connecting to two vertices, and each pointing into different cardinal directions.\label{FourCornerLemma}
\end{theorem}
\subsubsection*{Proof of Theorem~\ref{FourCornerLemma}}
Following from theorem~\ref{NorthwestCornerLemma}, for square lattice with defines a boundary, there is one corner with an edge to its east and one to its south. 
\section{First Algorithmic Approach}
By assuming that the westernmost vertex of the northernmost vertices always has exactly two edges rightward and downward, 